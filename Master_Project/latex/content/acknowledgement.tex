% !TEX root = ../thesis-example.tex
%
\pdfbookmark[0]{Acknowledgement}{Acknowledgement}
\chapter*{Remerciements}
\label{sec:acknowledgement}
\vspace*{-10mm}

J'ai choisi ma langue natale pour faire mes remerciements afin que toutes celles et ceux qui m'ont soutenu dans mon parcours puissent recevoir ma gratitude à sa juste valeur.

Je tiens, pour commencer, à remercier  Christophe Ansermoz, Emmanuel Duterme et Gregory Chevalley de la banque Pictet pour avoir cru en moi et m'avoir fait confiance pour exercer l'art, que j'ai particulièrement apprécié développer, de l'évaluation de produits financiers. 

Il s'ensuit naturellement les remerciements les plus sincères envers mes professeurs Fabio Nobile et Julien Hugonnier pour avoir accepté de superviser ce projet qui m'est cher. En plus de leur aide, je suis très reconnaissant de la collaboration précieuse que j'ai entretenue avec mon assistant, co-superviseur, Francesco Statti. 

Mais au delà de ce projet de master, c'est un parcours académique très rude qui s'achève, et c'est pourquoi je tiens à réserver tout particulièrement le mot de la fin à ma compagne, Maria Pacios, qui m'a soutenu, encouragé et surtout permis de surpasser toutes les difficultés que j'ai pu rencontrer. Je ne pourrais jamais être assez reconnaissant pour tout ce qu'elle a pu m'apporter durant cette aventure, qui a aussi, je l'admets, été éprouvante pour elle. J'espère que cette place d'honneur puisse remplacer tous les mots nécessaires pour exprimer ma reconnaissance infinie.

Finalement, si des collègues, des amis, des proches, mes parents ou mes beaux-parents sont amenés à lire ce projet, je ne les oublie pas et tiens également à les remercier pour avoir partagé cette période de vie à l'EPFL ou même une partie avec moi.

Bonne lecture!

\hfill \textit{Je dédie ce travail à \MVAt baboOu\_\_}

\hfill Valentin Bandelier 
